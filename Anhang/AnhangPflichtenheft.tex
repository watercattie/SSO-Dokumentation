\newpage
\subsection{Pflichtenheft (Auszug)}
\label{app:Pflichtenheft}

Die geplante Umsetzung der im Lastenheft (Auszug siehe \ref{app:Lastenheft}) definierten Anforderungen wird
in folgendem Auszug aus dem Pflichtenheft beschrieben:

\subsubsection*{Umsetzung der Anforderungen}

\begin{enumerate}[itemsep=0em,partopsep=0em,parsep=0em,topsep=0em]

\item Single-Sign-On-Server
	\begin{enumerate}
		\item Für das speichern der Benutzerdaten wird die firmeninterne MySQL Datenbank genutzt. 
		\item Innerhalb des \acs{SSO}-Servers gibt eine View für die Benutzerverwaltung:	
		\begin{itemize}
			\item Admins können Nutzer anlegen, löschen, bearbeiten
			\item Benutzer können nur ihr eigenes Profil löschen und bearbeiten
		\end{itemize}
	\item Als Authentifizierungsprotokoll wird OAuth2 \footnote{vgl. \cite{OAuth2}}genutzt.
	\begin{itemize}
		\item die Autorisierung wird hierbei über die API vorgenommen
		\item die Authentifizierung erfolgt ausschließlich über den Server
		\item zur Umsetzung wird das Laravel Paket Passport\footnote{vgl. \cite{Passport}} genutzt
	\end{itemize}
	\end{enumerate}

\item Login Prozess der Client Anwendung
		\begin{enumerate}
		\item Für das Speichern der Benutzerdaten wird die firmeninterne MySQL Datenbank genutzt. 
		\item Die wy-connect Anbindung an andere Laravel Anwendungen wird mittels Installation eines bereitgestellten Pakets umgesetzt.
		\begin{itemize}
		\item dieses Paket kann über einen einfachen Konsolenbefehl in den Clienten integriert werden
		\item der wy-connect Provider liegt dabei auf der firmeneigenen Gitlab Instanz
		\item zur Umsetzung wird das Laravel Paket Socialite\footnote{vgl. \cite{Socialite}} genutzt
	\end{itemize}
		\item Nachdem Login Prozess am \acs{SSO} wird der Benutzer gefragt ob der Test Client auf die Nutzerdaten zugreifen darf.
		\item Innerhalb des Login Prozesses am Clienten werden die Nutzerdatzen nur vom \acs{SSO}-Server abgefragt.
		\end{enumerate}
\item Sonstige Anforderungen
		\begin{enumerate}
		\item Über das Webinterface kann sich ein Nutzer anmelden und hat über eine übersichtlich gestaltete  \acs{GUI} Zugriff auf seine Einstellungen.
		\item Das Programm läuft als Webanwendung.
		\item Zur Versionskontrolle wird der firmeneigene GitLab-Server\footnote{vgl. \cite{GitLab}} genutzt.\\
		{[...]}
		\end{enumerate}
\end{enumerate}

