\newpage
\subsection{Lastenheft (Auszug)}
\label{app:Lastenheft}
Es folgt ein Auszug aus dem Lastenheft mit Fokus auf die Anforderungen:

Die Anwendung muss folgende Anforderungen erfüllen: 
\begin{enumerate}[itemsep=0em,partopsep=0em,parsep=0em,topsep=0em]
\item Single-Sign-On-Server
	\begin{enumerate}
	\item Die Anwendung verfügt über eine eigene Datenbank für die Benutzerdaten.
	\item Anzeigen einer Übersichtsseite für autorisierte Clienten und Token mit allen relevanten Informationen zu diesen.
	\item Die Authentifizierung muss über einen sicheren Kanal erfolgen.
	\end{enumerate}
\item Login Prozess der Client Anwendung
	\begin{enumerate}
	\item Die Client Anwendung verfügt über eine eigene Benutzerdatenbank.
	\item Die Authentifizierungsmöglichkeit über wy-connect muss einfach auf andere Clients übertragbar sein.
	\item Der Nutzer muss die Client Anwendung zum Zugriff autorisieren.
	\item Der Nutzer meldet sich nur am Server mit seinen Nutzerdaten an.
	\end{enumerate}
\item Sonstige Anforderungen
	\begin{enumerate}
	\item Der Server sowie der Login Prozess am Client soll über eine \acs{GUI} intuitiv bedient werden können.
	\item Der Server sowie der Client müssen ohne Installation von zusätzlicher Software über einen
Webbrowser im Intranet erreichbar sein. 
	\item Zur Versionskontrolle der Anwendungsentwicklung soll ein Git\footnote{vgl. \cite{Git}}-Repository
verwendet werden.
	\item Die Anwendung soll in der Programmiersprache \acs{PHP}\footnote{vgl. \cite{PHP}} mittels des Frameworks Laravel\footnote{vgl. \cite{Laravel}} umgesetzt werden.	
	\item Der Server soll auf dem firmeninternen Webserver gehostet
werden.
	\item Bei Einsatz einer MySQL-Datenbank\footnote{vgl. \cite{MySQL}} soll der firmeninterne MySQL Server Verwendung
finden.
	\item Der Einrichtungsprozess für neue Clienten muss dokumentiert werden.
	\end{enumerate}
\end{enumerate}
		{[...]}


