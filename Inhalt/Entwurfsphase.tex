% !TEX root = ../Projektdokumentation.tex
\section{Entwurfsphase} 
\label{sec:Entwurfsphase}

\subsection{Zielplattform}
\label{sec:Zielplattform}

Wie in Abschnitt \fullnameref{sec:Projektziel} erwähnt, soll am Ende des Abschlussprojektes eine eigenständige Webanwendung vorliegen. Das Deployment und die Aktualisierung wird somit erleichtert.
Die Daten, auf die zugegriffen werden soll, sind in einer bestehenden MySQL Datenbank gespeichert. 

Als Programmiersprache wurde \acs{PHP}\footnote{vgl. \cite{PHP}} gewählt. Dies bot sich an, da viele bestehende Anwendungen bereits in PHP geschrieben sind und sie sich damit leichter an wy-connect anbinden lassen und die Entwickler wycomcos mit dieser Sprache vertraut sind. 
Wie auch die bestehenden Webapplikationen wird bei der Implementation das Framework Laravel\footnote{vgl. \cite{Laravel}} genutzt.

\subsection{Framework}
\label{sec:Framework}

Laravel ist ein freies \acs{PHP}-Webframework, welches dem \acs{MVC}-Muster folgt. 

Es ermöglicht neben dem im folgenden unter \ref{sec:Architekturdesign} erläuterten MVC-Architekturdesign, \acs{REST}-Webdienste zu implementieren.

Laravel wird mit dem \acs{ORM} Eloquent und einem gut bedienbaren Migrationssystem ausgeliefert. Damit werden Objekte einer objektorientierten Anwendung in eine relationale Datenbank überführt.

Das Framework bringt von Hause aus ein Authentifizierungspaket, was auch von wyconnect genutzt wird und durch die Pakete Socialite und Passport ergänzt wird um den OAuth2 Mechanismus zu implementieren. Dazu mehr im Kapitel \fullnameref{sec:Architekturdesign}.

\subsection{Architekturdesign}
\label{sec:Architekturdesign}

Beim Design wurde dem MVC Muster gefolgt, welches Laravel schon mit sich bringt. Die Anwendung wird in 3 Komponenten aufgespalten und sicher damit Flexibilität, Anpassbarkeit und Wiederverwendbarkeit.
Bei einer späteren Implementierung als native Anwendung, kann das Model beibehalten werden und nur die View und der Controller müssten teilweise umgeschrieben werden.

Model
Das Model enthält Daten zur Weiterverarbeitung. In vielen Fällen spiegelt ein Model
eine Tabelle in der Datenbank wieder, so auch beim Eloquent ORM von Laravel. Nur in dieser Model Klasse werden die Daten bearbeitet oder erfasst. Eine Trennung vom Controller ist erwünscht.

View
Die View ist das Benutzerinterface. Daten von Model und Controller werden visualisiert und leitet \zB Benutzeraktionen und Formulare weiter. Innerhalb der View sollte ein Zugriff auf das Model oder den Controller vermieden werden. Somit bleiben alle Daten statisch und werden nicht verändert.

Controller
Der Controller empfängt Anfragen (Requests) von der View. In diesen Request sind \bspw die Login Daten eines Benutzers. Der Controller verarbeitet die Daten und sendet eine Anfrage an das User Model, welches dem Controller den richten User zurückgibt. Dieser loggt den Benutzer ein und die View zeigt den erfolgreichen Login Prozess an. 



\subsection{Entwurf der Benutzeroberfläche}
\label{sec:Benutzeroberflaeche} 

Das Laravel Passport Paket bringt eine beispielhafte Oberfläche mit, in der alle relevanten Anforderungen Anwendung finden. 
Dazu sind Screenshots und Mockups auf  \fullnameref{app:wyconnect-mockup} zu sehen. Timy stellt hier einen Client dar. Dabei handelt es sich um eine Zeiterfassungssoftware, die die Autorin im Rahmen ihrer Ausbildung implementiert hat und die auch über Single-Sign-On laufen soll. 

Über die Hauptansicht kann der Administrator die autorisierten Clients verwalten, die per \acs{SSO} auf  Benutzerdaten zugreifen dürfen. Benutzer können lediglich ihr eigenes Profil verwalten.

\subsection{Datenmodell}
\label{sec:Datenmodell}

Im folgenden sollen die wichtigsten Komponenten des Datenmodells genannt und kurz erläutert werden. 
Das Laravel Passport Paket erleichtert die Aufgabe der Autorin des Datenmodells immens. Mit Installation des Pakets werden automatisch alle Tabellen mit dem Konsolenbefehl \Code{php artisan migrate} migriert. im \fullnameref{app:ERM} werden die von der Autorin benötigten Entitäten dargestellt. 

\tabelle{Entitaeten}{tab:Entitaeten}{Entitaeten.tex}

** noch erweiterbar mit Details **

\subsection{Geschäftslogik}
\label{sec:Geschaeftslogik}


\subsection{Deployment}
\label{sec:Deployment}

\subsection{Pflichtenheft/Datenverarbeitungskonzept}
\label{sec:Pflichtenheft}
\begin{itemize}
	\item Auszüge aus dem Pflichtenheft/Datenverarbeitungskonzept, wenn es im Rahmen des Projekts erstellt wurde.
\end{itemize}

\paragraph{Beispiel}
Ein Beispiel für das auf dem Lastenheft (siehe Kapitel~\ref{sec:Lastenheft}: \nameref{sec:Lastenheft}) aufbauende Pflichtenheft ist im \Anhang{app:Pflichtenheft} zu finden.


\Zwischenstand{Entwurfsphase}{Entwurf}
