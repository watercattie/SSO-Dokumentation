% !TEX root = ../Projektdokumentation.tex
\section{Fazit} 
\label{sec:Fazit}

\subsection{Soll-/Ist-Vergleich}
\label{sec:SollIstVergleich}
Rückblickend betrachtet kann festgehalten werden, dass alle Anforderungen gemäß dem Pflichtenheft erfüllt wurden. 
In Tabelle~\ref{tab:Vergleich} wird die Zeit, die tatsächlich für die einzelnen Phasen benötigt wurde, der zuvor eingeplanten Zeit gegenübergestellt. 
Es ist zu erkennen, dass nur geringfügig von der Zeitplanung abgewichen wurde. Die sich daraus ergebenen Differenzen konnten untereinander kompensiert werden, sodass das Projekt in dem von der IHK festgelegten Zeitrahmen von 70 Stunden umgesetzt werden konnte.
Die zeitlichen Abweichungen kamen auf Grund von Mehraufwand bei der Recherche des OAuth2 Protokolls und gemindertem Aufwand bei der Implementierung auf Grund des verwendeten Frameworks zu Stande.
\tabelle{Soll-/Ist-Vergleich}{tab:Vergleich}{Zeitnachher.tex}

\subsection{Lessons Learned}
\label{sec:LessonsLearned}
Im Zuge der Projektdurchführung konnte die Autorin umfangreiche Erfahrungen auf dem Gebiet der Planung und Umsetzung sammeln.
Besonders deutlich wurde, wie wichtig eine gute Recherche ist, da ohne gründliches Wissen über das verwendete OAuth2 Protokoll keine Implementierung möglich gewesen wäre.
Auch das verwendete Framework Laravel und seine Pakete erleichterten zwar die eigentliche Implementierung, aber ohne Verständnis und vorherige Recherche kann dieses Werkzeug nicht bedient werden. 
\subsection{Ausblick}
\label{sec:Ausblick}
Das Projekt findet aktuell Anwendung in den ersten Clientanwendungen und soll künftig noch auf Dienste ausgeweitet werden, die nicht mit Laravel geschrieben wurden und somit noch angepasst werden müssen. Auch ist es denkbar die Sicherheit zu erhöhen und eine Zwei-Faktor-Authentifizierung oder eine Anbindung an die AD einzuführen.
