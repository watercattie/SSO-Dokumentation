% !TEX root = ../Projektdokumentation.tex
\section{Dokumentation}
\label{sec:Dokumentation}
Die Anwendung muss aufgrund ihrer einfachen Bedienbarkeit für den Nutzer nicht dokumentiert
sein. Die Programmführung ist intuitiv und selbsterklärend.
Für die Entwickler wurden aussagekräftige Methoden und Klassennamen verwendet, sowie
sich an die Framework-Spezifikationen gehalten und das MVC-Modell strikt umgesetzt.
Für den Code selbst (Variablen-, Methodennamen, \etc) wurden selbsterklärende
englische Begriffe verwendet, damit dieser im Clean Code-Prinzip (auch ohne
Kommentare) lesbar ist.
Mit der kostenlosen Software PHPDoc\footnote{Vgl. \cite{phpdoc}}  wurde eine Entwicklerdokumentation automatisch generiert. Bei dieser Entwicklerdokumentation handelt es sich um eine detaillierte Beschreibung der Klassen, die in der Anwendung verwendet werden. Außerdem werden deren Attribute und Methoden sowie die Abhängigkeiten der Klassen untereinander erläutert. Diese Dokumentation soll dem Entwickler als Übersicht und Nachschlagewerk dienen. Im \Anhang{app:phpdoc} findet sich eine Abbildung dazu.

\Zwischenstand{Dokumentation}{Dokumentation}
