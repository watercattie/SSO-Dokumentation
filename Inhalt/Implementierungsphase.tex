% !TEX root = ../Projektdokumentation.tex
\section{Implementierungsphase} 
\label{sec:Implementierungsphase}

\subsection{Implementierung der Datenstrukturen}
\label{sec:ImplementierungDatenstrukturen}

Die zwei wichtigsten Datentypen, die das Herzstück der Anwendung bilden, sind Regeln und Kompositstrukturen, d.h. logische Verkettungen von Regeln.

Beide Datenstrukturen wurden als \ac{ADT} realisiert, genauer gesagt als \textit{Summentyp}.

Eine Regel kann zwei mögliche Formen annehmen. Eine \textit{SimpleRule}, d.h. eine einfache Regel, die nur aus Name, Beschreibung und dem auszuführenden \acs{SQL}-Query besteht, oder eine \textit{CompositeRule}, also eine zusammengesetzte Regel. Zusammgesetzte Regeln besitzen selbst keine \acs{SQL}-Abfrage, sonder enthalten stattdessen eine Verkettung von Regeln, wobei diese verketteten Regeln wiederum Kompositregeln sein können. Die implementierung des Regel \acs{ADT} befindet sich im \Anhang{app:RegelADT}.

Eine solche Verkettung von Regeln wird durch den \textit{Composition} Datentyp abgebildet. Ein Komposit mehrere Formen annehmen, welches die Art der Verkettung repräsentiert. Die ersten vier Formen sind eine \textbf{AND}, \textbf{OR}, \textbf{XOR} oder \textbf{NAND} Verknüpfung. Eine Komposit stellt eine Baumstruktur dar, wobei jeder Zweig des Baums wiederrum ein Komposit enthält. Die Blätter des Baumes enthalten den speziellen Typ \textbf{Value} der keine weitere Verkettung, sondern eine \textit{SimpleRule} enthält.

\paragraph{Beispiel}
Gegeben sei der folgende Ausdruck, wobei $S_n$ eine \textit{SimpleRule} darstellt
\begin{align*}
S_1 \wedge (S_2 \oplus S_3)
\end{align*}

Daraus würde sich also folgendes Komposit ergeben:

\Tree [.And [.Value $S_1$ ]
				   [.Xor [.Value $S_2$ ]
				  		    [.Value $S_3$ ]]]

Die Implementierung des \textit{Composition} Datentyps findet sich im \Anhang{app:CompositeADT}.

\subsection{Implementierung des Kommandozeileninterfaces}
\label{sec:ImplementierungKommandozeileninterface}

\begin{itemize}
	\item Scopt Library (erstellt automatisch Hilfetext)
	\item Kommandozeileninterface macht nicht mehr als den Runner mit der CLI Config aufzurufen
	\item Screenshot des Hilfetextes im Anhang
\end{itemize}


\subsection{Implementierung der Geschäftslogik}
\label{sec:ImplementierungGeschaeftslogik}

\begin{itemize}
	\item Beschreibung des Vorgehens bei der Umsetzung/Programmierung der entworfenen Anwendung.
	\item \Ggfs interessante Funktionen/Algorithmen im Detail vorstellen, verwendete Entwurfsmuster zeigen.
	\item Quelltextbeispiele zeigen.
	\item Hinweis: Wie in Kapitel~\ref{sec:Einleitung}: \nameref{sec:Einleitung} zitiert, wird nicht ein lauffähiges Programm bewertet, sondern die Projektdurchführung. Dennoch würde ich immer Quelltextausschnitte zeigen, da sonst Zweifel an der tatsächlichen Leistung des Prüflings aufkommen können.
\end{itemize}

\paragraph{Beispiel}
Die Klasse \texttt{Com\-par\-ed\-Na\-tu\-ral\-Mo\-dule\-In\-for\-ma\-tion} findet sich im \Anhang{app:CNMI}.  


\Zwischenstand{Implementierungsphase}{Implementierung}
