% !TEX root = ../Projektdokumentation.tex
\section{Analysephase} 
\label{sec:Analysephase}


\subsection{Ist-Analyse} 
\label{sec:IstAnalyse}

Derzeit gibt es bei der Anmeldung an verschiedene Dienste der wycomco GmbH keine Optimierungen. 
Bei allen Anwendungen kann man sich nicht selbständig registrieren und sich damit ein Benutzerkonto anlegen. Der jeweilige Administrator oder Befugte muss den User per Hand per Email einladen, damit dieser die Registrierung durchführen kann und nach einem Akivierungslink die App vollständig nutzbar ist. 

Dieser Rhythmus lässt wie folgt beschreiben:
\begin{enumerate}
	\item Anlegen eines vorläufigen Benutzerkontos durch den Admin
	\item Benutzer bekommt eine Mail und klickt auf den Registrierungslink
	\item Benutzer füllt das Registrierungsformular aus
	\item Benutzer bekommt eine Email mit einem Aktivierungslink
	\item Benutzer klickt auf den Link und lässt sich damit aktivieren
	\item Benutzer muss sich einloggen und kann damit die Anwendung vollständig nutzen
\end{enumerate}

Wie schnell ersichtlich wird, ist der Aufwand immens, wenn dieser Vorgang auch nur dreimal wiederholt werden muss.
Dieser Prozess ist sehr zeitaufwändig. Auch im täglichen Geschäft ist es erforderlich sich an jeder Anwendung einzeln anzumelden. Wie in Abschnitt \ref{sec:Projektbegruendung} schon erwähnt, führt dies schnell zu Trivialpasswörtern und damit zu einem Sicherheitsrisiko. 


\subsection{Wirtschaftlichkeitsanalyse}
\label{sec:Wirtschaftlichkeitsanalyse}

Durch den momentanen Prozess entsteht ein hoher zeitlicher Mehraufwand, der durch die Umsetzung des Projektes verringert werden kann.
Damit ist es dringend nötig dieses Projekt umzusetzen. Ob sich das auch wirtschaftlich begründen lässt, wird in den nächsten Abschnitten erläutert.


\subsubsection{\gqq{Make or Buy}-Entscheidung}
\label{sec:MakeOrBuyEntscheidung}

Da es sich bei der Problematik um ein sehr spezifisches Problem der Entwicklung von \textit{syn6} handelt, konnten die Entwickler kein geeignetes Tool auf dem Markt finden, was allen benötigten Ansprüchen entspricht. Darum wurde sich dazu entschieden, das Projekt in Eigenentwicklung durchzuführen.


\subsubsection{Projektkosten}
\label{sec:Projektkosten}

Bei der Berechnung der Projektkosten müssen sowohl die Personalkosten, die durch die Entwicklung des Projektes anfallen, als auch die verwendeten Ressourcen berücksichtigt werden. Bei den Personalkosten muss weiterhin zwischen den Stundensätzen von Auszubildenden und Mitarbeitern unterschieden werden.

Der Stundensatz eines Mitarbeiters wird mit \eur{37,50} bemessen, der eines Auszubildenden mit \eur{10}. Für die Nutzung der Ressourcen\footnote{Räumlichkeiten, Arbeitsplatz, \etc} wird ein pauschaler Stundensatz von \eur{15} angenommen. Sämtliche Angaben stammen aus dem Controlling.

Aus diesen Werten ergeben sich die Projektkosten in Tabelle~\ref{tab:Kostenaufstellung}.

\tabelle{Kostenaufstellung}{tab:Kostenaufstellung}{Kostenaufstellung.tex}


\subsubsection{Amortisationsdauer}
\label{sec:Amortisationsdauer}

Die Automatisierung des Validierungsprozesses hat eine deutliche Zeitersparnis zur Folge. 

Da es sich nicht um einen täglichen Prozess handelt, und die Dauer einer manuellen Validierung stark abhängig von der Komplexität der zu prüfenden Regel ist, ist es schwer eine Aussage über die tatsächliche Zeitersparnis zu treffen. Schätzungen der Entwickler lagen bei einer Einsparung von 15-60 Minuten pro Prüfung. Für die Berechnung der Amortisationsdauer wird davon ausgegangen, dass eine Prüfung einmal in der Woche nötig ist. Für die Zeitersparnis pro Prüfung werden als Mittelwert 37,5 Minuten angenommen. 

Dies ergibt eine tägliche Ersparnis von

\begin{eqnarray}
\frac{37,5 \mbox{ min/Woche}}{5 \mbox{ Tage/Woche}} = 7,5 \mbox{ min/Tag}
\end{eqnarray}

Bei einer Zeiteinsparung von 7,5 Minuten pro Tag für beide Backend Entwickler und 254 Arbeitstagen\footnote{vgl. http://www.schnelle-online.info/Arbeitstage/Anzahl-Arbeitstage-2015.html} im Jahr ergibt sich eine Zeiteinsparung von 
\begin{eqnarray}
2 \cdot 254 \frac{Tage}{Jahr} \cdot 7,5 \frac{min}{Tag} = 3810 \frac{min}{Jahr} \approx 63,5 \frac{h}{Jahr} 
\end{eqnarray}

Dadurch ergibt sich eine jährliche Einsparung von 
\begin{eqnarray}
63,5 \mbox{ h} \cdot \eur{(37,5 + 15)}{\mbox{/h}} = \eur{3333,75}
\end{eqnarray}

Die Amortisationszeit beträgt also $\frac{\eur{2065,00}}{\eur{3333,75}\mbox{/Jahr}} \approx 0,6 \mbox{ Jahre} \approx 7 \mbox{ Monate}$.


\subsection{Nutzwertanalyse}
\label{sec:Nutzwertanalyse}

Neben den in \fullnameref{sec:Amortisationsdauer} aufgeführten wirtschaftlichen Vorteilen ergeben sich durch Realisierung des Projekts noch einige zusätliche nicht-monitäre Vorteile.

Ein automatisierter Prüfprozess ermöglicht es, das Ergebnis einer Prüfung weiterzuvearbeiten, um es \zB in einer Weboberfläche oder einem Monitoring System anzeigen zu lassen. Außerdem ermöglicht es die Einbindung in das \acs{CI}-System um die Datenbank kontinuierlich zu Prüfen. 

Obwohl das Projekt im Rahmen von \textit{syn6} entwickelt wurde, lässt es sich auch Problemlos auf andere Projekte und Datenbanken Anwenden. Der tatsächliche Nutzen geht also über \textit{syn6} hinaus.


\subsection{Qualitätsanforderungen}
\label{sec:Qualitaetsanforderungen}

Die Qualitätsanforderungen an die Anwendung lassen sich Tabelle~\ref{tab:Qualitaetsanforderungen} entnehmen.

\tabelle{Qualitätsanforderungen}{tab:Qualitaetsanforderungen}{Qualitaetsanforderungen.tex}


\subsection{Lastenheft/Fachkonzept}
\label{sec:Lastenheft}
\begin{itemize}
	\item Auszüge aus dem Lastenheft/Fachkonzept, wenn es im Rahmen des Projekts erstellt wurde.
	\item Mögliche Inhalte: Funktionen des Programms (Muss/Soll/Wunsch), User Stories, Benutzerrollen
\end{itemize}

\paragraph{Beispiel}
Ein Beispiel für ein Lastenheft findet sich im \Anhang{app:Lastenheft}. 

\Zwischenstand{Analysephase}{Analyse}
