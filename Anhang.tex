% !TEX root = Projektdokumentation.tex
\section{Anhang}
\subsection{Detaillierte Zeitplanung}
\label{app:Zeitplanung}

\tabelleAnhang{ZeitplanungKomplett}

\subsection{Verwendete Resourcen}
\label{app:Ressourcen}

\subsubsection{Hardware}

\begin{itemize}
	\item Büroarbeitsplatz mit privatem MacBook Pro
\end{itemize}

\subsubsection{Software}

\begin{itemize}
	\item OS X 10.11 El Capitan – Betriebssystem
	\item StarUML – Anwendung zum Erstellen von \acs{UML}-Diagrammen
	\item Scala – Programmiersprache
	\item IntelliJ Community Edition – Entwicklungsumgebung Scala
	\item git – Verteilte Versionsverwaltung
	\item Gitlab – Selfhosted Repository Verwaltung
	\item texmaker – \LaTeX\xspace Editor
	\item neovim – Moderne Distribution des Editors \textit{vim}
	\item Postgresql – Datenbanksystem
	\item pgAdmin – Verwaltungswerkzeug für Postgres Datenbanken
\end{itemize}

\newpage
\subsection{Lastenheft (Auszug)}
\label{app:Lastenheft}
Es folgt ein Auszug aus dem Lastenheft mit Fokus auf die Anforderungen:

Die Anwendung muss folgende Anforderungen erfüllen: 
\begin{enumerate}[itemsep=0em,partopsep=0em,parsep=0em,topsep=0em]
\item Single-Sign-On-Server
	\begin{enumerate}
	\item Die Anwendung verfügt über eine eigene Datenbank für die Benutzerdaten.
	\item Anzeigen einer Übersichtsseite für autorisierte Clienten und Token mit allen relevanten Informationen zu diesen.
	\item Die Authentifizierung muss über einen sicheren Kanal erfolgen.
	\end{enumerate}
\item Login Prozess der Client Anwendung
	\begin{enumerate}
	\item Die Client Anwendung verfügt über eine eigene Benutzerdatenbank.
	\item Die Authentifizierungsmöglichkeit über wy-connect muss einfach auf andere Clients übertragbar sein.
	\item Der Nutzer muss die Client Anwendung zum Zugriff autorisieren.
	\item Der Nutzer meldet sich nur am Server mit seinen Nutzerdaten an.
	\end{enumerate}
\item Sonstige Anforderungen
	\begin{enumerate}
	\item Der Server sowie der Login Prozess am Client soll über eine \acs{GUI} intuitiv bedient werden können.
	\item Der Server sowie der Client müssen ohne Installation von zusätzlicher Software über einen
Webbrowser im Intranet erreichbar sein. 
	\item Zur Versionskontrolle der Anwendungsentwicklung soll ein Git\footnote{vgl. \cite{Git}}-Repository
verwendet werden.
	\item Die Anwendung soll in der Programmiersprache \acs{PHP}\footnote{vgl. \cite{PHP}} mittels des Frameworks Laravel\footnote{vgl. \cite{Laravel}} umgesetzt werden.	
	\item Der Server soll auf dem firmeninternen Webserver gehostet
werden.
	\item Bei Einsatz einer MySQL-Datenbank\footnote{vgl. \cite{MySQL}} soll der firmeninterne MySQL Server Verwendung
finden.
	\item Der Einrichtungsprozess für neue Clienten muss dokumentiert werden.
	\end{enumerate}
\end{enumerate}
		{[...]}



\clearpage

\subsection{Use-Case-Diagramm}
\label{app:Use-Case-Diagramm}
Das folgende Diagramm beschreibt den Anmeldeprozess an verschiedenen Systemen ohne \acs{SSO}.
\begin{figure}[htb]
\centering
\includegraphicsKeepAspectRatio{use_case_diagramm.pdf}{0.7}
\caption{Use-Case-Diagramm}
\end{figure}
\clearpage

\subsection{Amortisationsdiagramm}
\label{app:Amortisationsdiagramm}
Das folgende Diagramm zeigt die Amortisation
\begin{figure}[htb]
\centering
\includegraphicsKeepAspectRatio{amortisation.png}{0.9}
\caption{Amortisations-Diagramm}
\end{figure}
\clearpage

\subsection{Kommandozeileninterface}
\label{app:cli}
\begin{figure}[htb]
\centering
\includegraphicsKeepAspectRatio{cli.png}{1}
\caption{Kommandozeileninterface}
\end{figure}
\clearpage

\subsection{Komponentendiagramm}
\label{app:Komponentendiagramm}
\begin{figure}[htb]
\centering
\includegraphicsKeepAspectRatio{Komponentendiagramm.png}{1}
\caption{Komponentendiagramm}
\end{figure}
\clearpage

\subsection{Aktivitätsdiagramm - Einlesen einer Regel}
\label{app:AktivitaetRegelEinlesen}
Das folgende Diagramm beschreibt das Einlesen einer Regel aus einer Regeldatei.
\begin{figure}[htb]
\centering
\includegraphicsKeepAspectRatio{AktivitaetRegelEinlesen.png}{0.9}
\caption{Einlesen einer Regel}
\end{figure}
\clearpage

\subsection{Beispiel Hauptregeldatei}
\label{app:MasterDatei}
Ein Beispiel einer Hauptdatei. In der Hauptdatei werden sowohl die Datenbankverbindungsparamer beschrieben als auch sämtliche Regeln aufgelistet, die evaluiert werden sollen.
\lstinputlisting{Listings/master.conf}

\subsection{Implementierung des Regel \acs{ADT}s}
\label{app:RegelADT}
\lstinputlisting[language=scala]{Listings/Rule.scala}
\clearpage

\subsection{Implementierung des Komposit \acs{ADT}s}
\label{app:CompositeADT}
\lstinputlisting[language=scala]{Listings/Composition.scala}

\newpage
\subsection{Pflichtenheft (Auszug)}
\label{app:Pflichtenheft}

Die geplante Umsetzung der im Lastenheft (Auszug siehe \ref{app:Lastenheft}) definierten Anforderungen wird
in folgendem Auszug aus dem Pflichtenheft beschrieben:

\subsubsection*{Umsetzung der Anforderungen}

\begin{enumerate}[itemsep=0em,partopsep=0em,parsep=0em,topsep=0em]

\item Single-Sign-On-Server
	\begin{enumerate}
		\item Für das speichern der Benutzerdaten wird die firmeninterne MySQL Datenbank genutzt. 
		\item Innerhalb des \acs{SSO-Servers} gibt eine View für die Benutzerverwaltung:	
		\begin{itemize}
			\item Admins können Nutzer anlegen, löschen, bearbeiten
			\item Benutzer können nur ihr eigenes Profil löschen und bearbeiten
		\end{itemize}
	\item Als Authentifizierungsprotokoll wird OAuth2 \footnote{vgl. \cite{OAuth2}}genutzt.
	\begin{itemize}
		\item die Autorisierung wird hierbei über die API vorgenommen
		\item die Authentifizierung erfolgt ausschließlich über den Server
		\item zur Umsetzung wird das Laravel Paket Passport\footnote{vgl. \cite{Passport}} genutzt
	\end{itemize}
	\end{enumerate}

\item Login Prozess der Client Anwendung
		\begin{enumerate}
		\item Für das speichern der Benutzerdaten wird die firmeninterne MySQL Datenbank genutzt. 
		\item Die wy-connect Anbindung an andere Laravel Anwendungen wird mittels Installation eines bereitgestellten Pakets umgesetzt.
		\begin{itemize}
		\item dieses Paket kann über einen einfachen Konsolenbefehl in den Clienten integriert werden
		\item der wy-connect Provider liegt dabei auf der firmeneigenen Gitlab Instanz
		\item zur Umsetzung wird das Laravel Paket Socialite\footnote{vgl. \cite{Socialite}} genutzt
	\end{itemize}
		\item Nachdem Login Prozess am \acs{SSO} wird der Benutzer gefragt ob der Test Client auf die Nutzerdaten zugreifen darf.
		\item Innerhalb des Login Prozesses am Clienten werden die Nutzerdatzen nur vom \acs{SSO}-Server abgefragt.
		\end{enumerate}
\item Sonstige Anforderungen
		\begin{enumerate}
		\item Über das Webinterface kann sich ein Nutzer anmelden und hat über eine übersichtlich gestaltete  \acs{GUI} Zugriff auf seine Einstellungen.
		\item Das Programm läuft als Webanwendung.
		\item Zur Versionskontrolle wird der firmeneigene GitLab-Server\footnote{vgl. \cite{GitLab}} genutzt.\\
		{[...]}
		\end{enumerate}
\end{enumerate}


